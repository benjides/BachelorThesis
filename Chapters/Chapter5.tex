% Chapter 4

\chapter{Conclusions}

\label{conclusions}
%----------------------------------------------------------------------------------------
As seen in the \ref{test stage} the performance obtained by all the proposed models is not positive enough to deploy these models in a production environment.

During this whole thesis development project, several approaches and iterations have been explored to improve the efficiency of the overall model with unsatisfactory outputs.

Based on the state of the art \ref{stateofart} analysis different papers seems to reach poor performance with disappointing results.

It can be concluded that song deep features combined with MLP models are not a suitable option and approach to label musical pieces. 

As already mentioned labeling genres is a complex and ongoing task that still needs development and further investigation, really subtle details can determine whether a song belongs to a genre or another. This task is even complicated for humans too that sometimes can not agree to which genre a song fits to. 

When does a song be different enough to constitute a genre itself?

Is it a sub-genre or a completely new one?

Since the genres are labels made up by humans to classify songs according to some diffused rules, the genre itself may not be present in the intrinsic features of the song and the labeling task is still a debatable task only humans themselves can decide.

\section{Achieved goals}
The primary goal of creating an Artificial Neural Network capable of labeling songs has been achieved.

This Network is able to label with more than one genre to a song constituting a Multi-label Multi-Class classification problem.

The model can label genres hierarchically creating several classifiers recursively complying with the established genre ontology.

\section{Pending goals}
Improve the model performance and surpass the outcome defined in the state of the art for the same problem.

\section{Further work}
To properly label song that is the main task of this paper some different alternatives can be chosen:

\begin{itemize}
	\item Use different architectures. Play with different MLP models and architectures. Using a Convolutional Neural Network along with the song spectrogram is a promising task that can produce better conclusions.
	\item Use different features. Combine and test different song features. Due to the limitations of the provided dataset some extra features that might have been useful could not be used.
	\item Explore different algorithms. Maybe the MLP is not a suitable solution to tackle the genre classification problem. Using alternatives could improve the performance like Support Vector Machines o Nearest Neighbors to name a few.
	\item Combining different ML techniques might improve the results, appointing the previous point, the use of several algorithms for small and highly coupled tasks can lead to superior results.
\end{itemize}
